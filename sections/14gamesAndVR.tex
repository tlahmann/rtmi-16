\todo[inline]{existing game research?? own section?}

\section{Games and the Virtual Reality}

\todo[inline]{JF: Die Wahl wirkt trotzdem ein bisschen willkürlich

Außerdem gibts The Lab nur für die Vive und nicht "regulär", Dota2 für VR ist auch nicht wirklich vergleichbar da das nur der Spectator Mode ist, Minecraft passt ganz gut als Vergleich aber da ist es auch noch unterschiedlich: Für die Vive gibt es nur Hacks, aber die Steuerung wird natürlich gut ausgenutzt. Gear und Oculus werden offiziell unterstützt aber da funktioniert die Interaktion anders als bei der Vive.

Zu FM 2017 hab ich jetzt nur Ankündigungen gefunden. Gibt's denn da was in VR?}

Desktop Monitor (P), Oculus rift (O), HTC Vive (V)

Possible games: Minecraft (AL, P/O), keep talking and nobody explodes (NL, P/O/H), project CARS GOTY edition (CL, P/O/H),  

\begin{table}[h]
	\caption{Popular games played in late 2016. The games offer a regular play mode, using the monitor, and a VR play mode with both the Oculus Rift and the HTC Vive}~\label{tab:popularGames}
	
	%{}\fontfamily{pcr}\selectfont
	\begin{tabular*}{\columnwidth}{ l | l | r | l }
		Gametitle & Genre\footnotemark & \parbox[c][2.4em][t]{2cm}{\begin{flushright}$\dfrac{Players}{Month}$(\footnotemark)\end{flushright}} & LM\footnotemark \\
		\hline
		Minecraft & RPG & 300 K & AL \\
		Keep Talking and Nobody \\Explodes & Puzzle & --- K & NL \\
		Project CARS & Racing & --- K & CL\\
	\end{tabular*}
	%}
	
\end{table}

\todo[inline]{table too small?}
\footnotetext{MOBA: Multiplayer Online Battle Arena; FPS: First-Person Shooter; RPG: Role-Playing Game}
\footnotetext{\url{http://steamcharts.com}, accessed Nov. 12., 2016}
\footnotetext{Types of Locomotion: Natural Locomotion (NL), Cockpit Locomotion (CL), Artificial Locomotion (AL)}
\todo[inline]{Die footnotes werden nicht richtig nummeriert... von hand fixen bei submission!?}


Write something about the motivation in games of Table~\ref{tab:popularGames}. Why are the games good or bad to play with VR and why the games are popular. Write about the movement of the player in the games. Write about the physics, the game principles and the differences to solely non-vr games.

\subsection{Popularity Of Games}
Since the beginning of virtual reality in late \textcolor{red}{20th century} there has been a huge potential to include the user in a virtual environment. Some systems have been developed in the past century that simplify some processes of work for different disciplines. 

Games have just lately adopted the potential for this immersive form of gaming and with the developement of the \textcolor{red}{\textit{Oculus Rift in 2014}}, the \textcolor{red}{\textit{HTC Vive in 2016}}, and the \textcolor{red}{\textit{Samsung Gear in 2015}} and the \textcolor{red}{damit einhergehend} distribution of these systems into the living room of many people.

In \textcolor{red}{gegensatz/ parallel dazu} the research on these fields has not made huge advances with the field.

PAPER on gaming addictions and popularity of games

Games using a virtual environment to show worlds to players give the player the feeling of being in the middle of the happening(?) but the game has very limited possibilities to tell storys with much varity, because the game can not change perspectives as easy. \textcolor{red}{wohingegen} games that do not use VR can handle attention themself by just moving the camera. these games can show views from other perspective without worriing about the user being confused too much. VR-Games need another way to handle interaction, i.e. the user has to hold some kind of controller interface to tell the system what he wants to do. Games outside of vr have similar interface problems but the process has been researched more than the interaction with VR-Games. \textcolor{red}{gleichzeitig} do vr-games \textcolor{red}{fordern} a much more active way of interaction, because the user has to complete different tasks as he is playing the game. this can be a negative point for people who would just rather enjoy a game than have a minor workout while playing.
			
pro vr: immerse, active, intuitive, innovative.
contra vr: active, expensive, hard to develop games

Why VR games are as popular as they are

\todo[inline]{JF: Auch oder vielleicht sogar mehr: Wieso nicht? Ist es nur dass die Technologie noch nicht verbreitet genug ist? Machen die Spiele was anders? Funktionieren traditionelle Spiele nicht einfach so in VR? etc.}

\subsection{Motivation In Games}
What Motivation can VR offer that other Games cannot offer. What motivation can normal games offer, that vr-games cannot offer.

motivation of players: (talk)\footnote{"Gamer Motivation Profile Findings - \#GamesUR US Conference 2016" Quantic Foundry Website, March, 25., 2015, accessed November 05., 2016, \url{http://quanticfoundry.com/2016/04/07/gdc-talk/}}
\begin{itemize}
	\item action
	\item social
	\item mastery
	\item achievement
	\item immersion
	\item creativity
	\item \textbf{age}
\end{itemize}

\subsection{Games Exclusively For VR}
Why games are developed solely for VR
