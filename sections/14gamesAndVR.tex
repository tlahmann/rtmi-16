%\todo[inline]{existing game research?? own section?}

\section{Games and the Virtual Reality}

\begin{figure}
	\centering
	\includegraphics[width=0.9\columnwidth]{./figures/placeholder}
	\caption[blabla]{Lorem Ipsum dolor sit amet bla bla bla, foo , bar bla blubb fooooooo bla bla bla bla bla bla bla}~\label{fig:foobar1}
\end{figure}

Virtual reality (VR) typically refers to computer technologies that use software to generate realistic images, sounds and other sensations that replicate a real environment (or create an imaginary setting), and simulate a user's physical presence in this environment, by enabling the user to interact with this space and any objects depicted therein using specialized display screens or projectors and other devices. VR has been defined as ''...a realistic and immersive simulation of a three-dimensional environment, created using interactive software and hardware, and experienced or controlled by movement of the body'' or as an ''immersive, interactive experience generated by a computer''.

To gain a brief insight into games that both support playing with Virtual Reality Headsets and without them and to further introduce research aspects the following section is intended to list a few differences, similarities and uniqueness within games. An overview these games can be found in table \ref{tab:popularGames}, where some properties of these games are listed. 

Starting with Minecraft. an open world sandbox game where the player has to mine blocks of different material and craft them into objects which he will need to complete the game task \todo[inline]{describe goal}

Another game is \textit{Keep Talking and Nobody Explodes} a puzzle game where one player has to describe and disarm a virtual explosive charge in a given time. his team members, who cannot see the bomb, have to explain what player one has to do in order to disarm the bomb and to save his and the team members lives.

Project CARS is a racing game for PC and other consoles. It offers wide VR support. Typical racing game.

Possible games: Minecraft (AL, P/O), keep talking and nobody explodes (NL, P/O/H), project CARS GOTY edition (CL, P/O/H),  

\begin{table}%[h]
	\caption{Popular games played in late 2016. The games offer a regular play mode, using the monitor, and a VR play mode with both the Oculus Rift and the HTC Vive}~\label{tab:popularGames}
	
	%{}\fontfamily{pcr}\selectfont
	\renewcommand{\arraystretch}{1.3}% for the vertical padding
	\begin{tabular*}{\columnwidth}{ p{33mm} l r l }
		Gametitle & Genre\footnotemark & \parbox[c][2.2em][t]{2cm}{\begin{flushright}$\dfrac{Players}{Month}$(\footnotemark)\end{flushright}} & LM\footnotemark \\
		\hline
		Minecraft & RPG & 300 K & ALM \\
		Keep Talking and \newline Nobody Explodes & Puzzle & 153.3 & NLM \\
		Project CARS & Racing & 1.01 K & CLM\\
	\end{tabular*}
	%}
	
\end{table}

\footnotetext[1]{RPG: Role-Playing Game}
\footnotetext[2]{\url{http://steamcharts.com}, accessed Jan. 3., 2017}
\footnotetext[3]{Types of Locomotion (LM): Natural Locomotion (NLM), Cockpit Locomotion (CLM), Artificial Locomotion (ALM)}
\todo[inline]{Die footnotes werden nicht richtig nummeriert... von hand fixen bei submission!?}


Write something about the motivation in games of Table~\ref{tab:popularGames}. Why are the games good or bad to play with VR and why the games are popular. Write about the movement of the player in the games. Write about the physics, the game principles and the differences to solely non-vr games.

\begin{figure}
	\centering
	\includegraphics[width=0.9\columnwidth]{./figures/placeholder}
	\caption[blabla]{Lorem Ipsum dolor sit amet bla bla bla, foo , bar bla blubb fooooooo bla bla bla bla bla bla bla}~\label{fig:foobar4}
\end{figure}

\subsection{Popularity Of Games}
Since the beginning of virtual reality in the 20th century there has been a huge potential to include the user in a virtual environment. Some systems have been developed in the past century that simplify some processes of work for different disciplines. 

Games have just lately adopted the potential for this immerse form of gaming and with the release of the \textit{Oculus Rift in March 2016}, the \textit{HTC Vive in April 2016}, the \textit{Samsung Gear VR in August 2015}, the \textit{Playstation VR in October 2015} or the \textit{Daydream View in November 2015} and the concomitant distribution of these systems into the living room of many people.

In contrast the research on these fields has not made huge advances with the field.

as in the Paper of Rejtö described 

Games using a virtual environment to show worlds to players give the player the feeling of being in the middle of the event but the game has very limited possibilities to tell stories with much variety, because the game can not change perspectives as easy. While games that do not use VR can handle attention themselves by just moving the camera. these games can show views from other perspective without worrying about the user being confused too much. VR-Games need another way to handle interaction, i.e. the user has to hold some kind of controller interface to tell the system what he wants to do. Games outside of vr have similar interface problems but the process has been researched more than the interaction with VR-Games. Simultaneously VR games require a much more active way of interaction, because the user has to complete different tasks as he is playing the game. This can be a negative point for people who would just rather enjoy a game than have a minor workout while playing.
			
pro vr: immerse, active, intuitive, innovative.
contra vr: active, expensive, hard(er) to develop games

Why VR games are as popular as they are

Why are VR games not as popular as regular games. Are the prices for the equipment too high or are the Games just not developed enough?

\todo[inline]{JF: Auch oder vielleicht sogar mehr: Wieso nicht? Ist es nur dass die Technologie noch nicht verbreitet genug ist? Machen die Spiele was anders? Funktionieren traditionelle Spiele nicht einfach so in VR? etc.}

\subsection{Motivation In Games}

\begin{figure}
	\centering
	\includegraphics[width=0.9\columnwidth]{./figures/placeholder}
	\caption[blabla]{Lorem Ipsum dolor sit amet bla bla bla, foo , bar bla blubb fooooooo bla bla bla bla bla bla bla}~\label{fig:foobar3}
\end{figure}


What Motivation can VR offer that other Games cannot offer. What motivation can normal games offer, that vr-games cannot offer.

motivation of players: (talk)\footnote{"Gamer Motivation Profile Findings - \#GamesUR US Conference 2016" Quantic Foundry Website, March, 25., 2015, accessed November 05., 2016, \url{http://quanticfoundry.com/2016/04/07/gdc-talk/}}
\begin{itemize}
	\item action - people who like the thrill and the challenge
	\item social - people who like to be together with others, socialize and help others
	\item mastery - -making complex decisions
	\item achievement - those who want to achieve something and show others that they are good, or those who want to compare to others
	\item immersion - those who want to (temporarily) escape the 'real' world and who want to feel immersed into a game
	\item creativity - those who want to be creative, build something .. making the game your own in any way. 
	\item \textbf{age} - differences between the young generation and the old one (older people mostly have another point of view on things and have other motivation) But also the younger generation, grown up with video games will have a different motivation than the 'older' generation today. (because of the constant companionship of games)
\end{itemize}

3 clusters 

action-social

mastery-achievement

immersion-creativity

\subsection{Games Exclusively For VR}
Why games are developed solely for VR

Is this section needed???
