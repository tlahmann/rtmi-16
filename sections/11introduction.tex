\section{Introduction}
\todo[inline]{write introduction}

The potential of Virtual Reality (VR) in gaming is an enormous factor both for the scientific and commercial world withing the next cycles of game development Users are now allowing the presence of more systems in their living room. Game developers need to adapt to the demand and depend on the research of competent people to improve their games further.

Many developers have already started to develop games that use the advances of VR and are serving quick, but the research has lacked behind in recent years. 

VR, although expensive, gains more and more interest and the audience grows almost daily. 
Games that do not naturally support VR are searching for ways to include VR in their games (DOTA, spectator mode)
Games, described in the section \textit{Games And VR} (\textcolor{red}{x-Ref erstellen})

Therefore an attempt will be made with this publication to show the many field in which research is necessary. 

I'm trying to evaluate and list the most interesting fields of study for the application of games in Virtual Reality (VR).

Most of the  points mentioned need to be looked into further as the development of VR games proceeds.

\todo[inline]{JF: Hier in die Introduction gehört dann auch noch wieso das Thema denn interessant ist. Du hast das ja schon im Abstract ein bisschen: Potential von VR groß, gerade auch im Bereich Games --> das kann mit einigen Beispielen sehr schön ausführen

ABER: es gibt noch kaum Forschung in dem Bereich, deshalb in dieser Arbeit dieser Überblick}

\todo[inline]{Beschreiben welche themen in den Kapitel erarbeitet werden... oder gehört sich das nicht -> Recherche}

%\begin{figure}
%	\centering
%	\includegraphics[width=0.9\columnwidth]{./figures/placeholder}
%	\caption{Foobar}~\label{fig:figure1}
%\end{figure}

%\textcolor{gray}{\blindtext[3]}
