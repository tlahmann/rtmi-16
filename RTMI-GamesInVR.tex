\documentclass[chi_draft]{sigchi}

% Use this command to override the default ACM copyright statement
% (e.g. for preprints).  Consult the conference website for the
% camera-ready copyright statement.

%% EXAMPLE BEGIN -- HOW TO OVERRIDE THE DEFAULT COPYRIGHT STRIP -- (July 22, 2013 - Paul Baumann)
% \toappear{Permission to make digital or hard copies of all or part of this work for personal or classroom use is      granted without fee provided that copies are not made or distributed for profit or commercial advantage and that copies bear this notice and the full citation on the first page. Copyrights for components of this work owned by others than ACM must be honored. Abstracting with credit is permitted. To copy otherwise, or republish, to post on servers or to redistribute to lists, requires prior specific permission and/or a fee. Request permissions from permissions@acm.org. \\
% {\emph{CHI'14}}, April 26--May 1, 2014, Toronto, Canada. \\
% Copyright \copyright~2014 ACM ISBN/14/04...\$15.00. \\
% DOI string from ACM form confirmation}
%% EXAMPLE END -- HOW TO OVERRIDE THE DEFAULT COPYRIGHT STRIP -- (July 22, 2013 - Paul Baumann)

% Arabic page numbers for submission.  Remove this line to eliminate
% page numbers for the camera ready copy
\pagenumbering{arabic}

% Load basic packages
\usepackage{balance}  % to better equalize the last page
%\usepackage{graphics} % for EPS, load graphicx instead 
\usepackage{graphicx}
\usepackage[labelfont={bf},textfont={bf},font=small]{caption}
\usepackage[labelfont={bf},textfont={bf},font=small]{subcaption} % if two images are rigth next to eachother
\usepackage[T1]{fontenc}
\usepackage{txfonts}
\usepackage{mathptmx}
\usepackage[pdftex]{hyperref}
\usepackage{color}
\usepackage{booktabs}
\usepackage{textcomp}
% Some optional stuff you might like/need.
\usepackage{microtype} % Improved Tracking and Kerning
% \usepackage[all]{hypcap}  % Fixes bug in hyperref caption linking
\usepackage{ccicons}  % Cite your images correctly!
\usepackage[utf8]{inputenc} % for a UTF8 editor only
\usepackage{tikz}
\newcommand*\circled[1]{\tikz[baseline=(char.base)]{
            \node[shape=circle,draw,inner sep=2pt] (char) {#1};}} % circled icons for image citation
						
% If you want to use todo notes, marginpars etc. during creation of your draft document, you
% have to enable the "chi_draft" option for the document class. To do this, change the very first
% line to: "\documentclass[chi_draft]{sigchi}". You can then place todo notes by using the "\todo{...}"
% command. Make sure to disable the draft option again before submitting your final document.
\usepackage[backgroundcolor=yellow,textsize=small]{todonotes}
\usepackage{blindtext}

% Paper metadata (use plain text, for PDF inclusion and later
% re-using, if desired).  Use \emtpyauthor when submitting for review
% so you remain anonymous.
\def\authorname{Tobias Lahmann}
\def\plaintitle{TITLE TITLE TITLE}
\def\plainauthor{\authorname}
\def\emptyauthor{}
\def\plainkeywords{Virtual Reality, psychology, research}
\def\plaingeneralterms{Documentation, Standardization}

% llt: Define a global style for URLs, rather that the default one
\makeatletter
\def\url@leostyle{%
  \@ifundefined{selectfont}{
    \def\UrlFont{\sf}
  }{
    \def\UrlFont{\small\bf\ttfamily}
  }}
\makeatother
\urlstyle{leo}

% To make various LaTeX processors do the right thing with page size.
\def\pprw{8.5in}
\def\pprh{11in}
\special{papersize=\pprw,\pprh}
\setlength{\paperwidth}{\pprw}
\setlength{\paperheight}{\pprh}
\setlength{\pdfpagewidth}{\pprw}
\setlength{\pdfpageheight}{\pprh}

% Make sure hyperref comes last of your loaded packages, to give it a
% fighting chance of not being over-written, since its job is to
% redefine many LaTeX commands.
\definecolor{linkColor}{RGB}{6,125,233}
\hypersetup{%
  pdftitle={\plaintitle},
  pdfauthor={\plainauthor},
%  pdfauthor={\emptyauthor},
  pdfkeywords={\plainkeywords},
  bookmarksnumbered,
  pdfstartview={FitH},
  colorlinks,
  citecolor=black,
  filecolor=black,
  linkcolor=black,
  urlcolor=linkColor,
  breaklinks=true,
}

% create a shortcut to typeset table headings
% \newcommand\tabhead[1]{\small\textbf{#1}}

% End of preamble. Here it comes the document.
\begin{document}

\title{\plaintitle}

\numberofauthors{1}
\author{
	\alignauthor{\authorname\\
		\affaddr{Research Trends in Media Informatics}\\
		\affaddr{Ulm, Germany}\\
		\email{tobias.lahmann@uni-ulm.de}}
%	\alignauthor{Julian Frommel\\
%		\affaddr{Research Trends in Media Informatics}\\
%		\affaddr{Ulm, Germany}\\
%		\email{julian.frommel@uni-ulm.de}}
}

\maketitle
\todo[inline]{need a title}

%----------------------------------------------------------------------------------------
%	Document
%----------------------------------------------------------------------------------------

\begin{abstract}
Virtual Reality techniques are suited very well for games since they incorporate users into the action and offer great potential for developers to extend the experience conveyed in their games. Understanding the advantages but also the disadvantages of VR in all fields of gaming is important for the future advancement of the field.
Many varying aspects and different field of needed research for \textit{Games in VR} will be examined here.
Other literature is often focusing either on the field of games and analyzing the impacts on players or the psychological affects of games and scenes, or is studying the field of virtual reality and the affects of this.
The objective of this publication is to combine both fields and show future development of Games in Virtual Reality.
\end{abstract}


\category{H.5.1}{[Multimedia Information Systems]: Artificial, augmented, and virtual realities}{}{}

\keywords{\plainkeywords}


\section{Introduction}
\todo[inline]{write introduction}

The potential of Virtual Reality (VR) in gaming is an enormous factor both for the scientific and commercial world withing the next cycles of game development Users are now allowing the presence of more systems in their living room. Game developers need to adapt to the demand and depend on the research of competent people to improve their games further.

Many developers have already started to develop games that use the advances of VR and are serving quick, but the research has lacked behind in recent years. 

VR, although expensive, gains more and more interest and the audience grows almost daily. 
Games that do not naturally support VR are searching for ways to include VR in their games (DOTA, spectator mode)
Games, described in the section \textit{Games And VR} (\textcolor{red}{x-Ref erstellen})

Therefore an attempt will be made with this publication to show the many field in which research is necessary. 

I'm trying to evaluate and list the most interesting fields of study for the application of games in Virtual Reality (VR).

Most of the  points mentioned need to be looked into further as the development of VR games proceeds.

\todo[inline]{JF: Hier in die Introduction gehört dann auch noch wieso das Thema denn interessant ist. Du hast das ja schon im Abstract ein bisschen: Potential von VR groß, gerade auch im Bereich Games --> das kann mit einigen Beispielen sehr schön ausführen

ABER: es gibt noch kaum Forschung in dem Bereich, deshalb in dieser Arbeit dieser Überblick}

\todo[inline]{Beschreiben welche themen in den Kapitel erarbeitet werden... oder gehört sich das nicht -> Recherche}

%\begin{figure}
%	\centering
%	\includegraphics[width=0.9\columnwidth]{./figures/placeholder}
%	\caption{Foobar}~\label{fig:figure1}
%\end{figure}

%\textcolor{gray}{\blindtext[3]}


\todo[inline]{existing game research?? own section?}

\section{Games and the Virtual Reality}

\todo[inline]{JF: Die Wahl wirkt trotzdem ein bisschen willkürlich

Außerdem gibts The Lab nur für die Vive und nicht "regulär", Dota2 für VR ist auch nicht wirklich vergleichbar da das nur der Spectator Mode ist, Minecraft passt ganz gut als Vergleich aber da ist es auch noch unterschiedlich: Für die Vive gibt es nur Hacks, aber die Steuerung wird natürlich gut ausgenutzt. Gear und Oculus werden offiziell unterstützt aber da funktioniert die Interaktion anders als bei der Vive.

Zu FM 2017 hab ich jetzt nur Ankündigungen gefunden. Gibt's denn da was in VR?}
\begin{table}[h]
	\caption{Popular games played in late 2016. The games offer a regular play mode, using the monitor, and a VR play mode. Games are chosen from different genres to compare the influence from and to virtual reality.}~\label{tab:popularGames}
	
	%{}\fontfamily{pcr}\selectfont
	\begin{tabular*}{\columnwidth}{ l l r r }
		Gametitle & Genre\footnotemark & \parbox[c][2.2em][t]{2cm}{\begin{flushright}$\dfrac{Players}{Month}$(\footnotemark)\end{flushright}} & Metascore\footnotemark \\
		\hline
		The Lab & Puzzle & 0.16 K & 74 \\
		Dota 2 & MOBA & 611 K & 90 \\
		%Counter Strike: GO & FPS & 330 & 83 \\
		Minecraft & RPG & 300 K & 93 \\
		Superhot (VR) & FPS & 0.19 K & 83 \\
		Football Mgr. '17 & Sports & 15 K & 79 \\
	\end{tabular*}
	%}
	
\end{table}

\todo[inline]{table too small?}
\footnotetext{MOBA: Multiplayer Online Battle Arena; FPS: First-Person Shooter; RPG: Role-Playing Game}
\footnotetext{\url{http://steamcharts.com}, accessed Nov. 12., 2016}
\footnotetext{\url{http://www.metacritic.com}, accessed Nov. 12., 2016}
\todo[inline]{Die footnotes werden nicht richtig nummeriert... von hand fixen bei submission!?}


Write something about the motivation in games of Table~\ref{tab:popularGames}. Why are the games good or bad to play with VR and why the games are popular. Write about the movement of the player in the games. Write about the physics, the game principles and the differences to solely non-vr games.

\subsection{Popularity Of Games}
Why VR games are as popular as they are

\subsection{Motivation In Games}
What Motivation can VR offer that other Games cannot offer

\todo[inline]{JF: Auch oder vielleicht sogar mehr: Wieso nicht? Ist es nur dass die Technologie noch nicht verbreitet genug ist? Machen die Spiele was anders? Funktionieren traditionelle Spiele nicht einfach so in VR? etc.}

motivation of players: (talk)\footnote{"Gamer Motivation Profile Findings - \#GamesUR US Conference 2016" Quantic Foundry Website, March, 25., 2015, accessed November 05., 2016, \url{http://quanticfoundry.com/2016/04/07/gdc-talk/}}
\begin{itemize}
	\item action
	\item social
	\item mastery
	\item achievement
	\item immersion
	\item creativity
	\item \textbf{age}
\end{itemize}

\subsection{Games Exclusively For VR}
Why games are developed solely for VR


\section{Further research trends}
\subsection{Physical And Mental Aspects}

\subsection{Physical Objects}

\begin{figure}
	\centering
	\includegraphics[width=0.9\columnwidth]{./figures/portallabrattest}
	\caption[Portal 2 : Lab Rat Panel Test]{Screenshot from the promotional video 'Portal 2 : Lab Rat Panel Test' (\ccbyncsa) of the game \textit{Portal 2 \textregistered\textcopyright} showing the general setup of floor panels mounted to robotic arms. This method enables developers to modify the structure of the floor at runtime.\footnotemark}~\label{fig:portallabrattest}
\end{figure}
\footnotetext{\textcopyright "Valve Software", [Online; accessed November 05., 2016],[Digitally revised] \url{https://www.youtube.com/watch?v=S7vFxs0ycn0}}



\section{Discussion}
\textcolor{gray}{\blindtext[5]}


\section{Summary}
All of the senses of the human body can be reached with todays technology. Some people may like the thought of totally immersed gaming, other find it disturbing.

Although a controversy appears in some fields the research will surely go on in the next years bringing many innovative and tense devices to our lives including better interaction, conclusive visual effects, stunning audio and astonishing touchable illusions. 

There are just many ways of finding new ways to improve VR gaming in the future. VR is on the edge of becoming a main way to experience games and the enormous potential offers just too many aspects to consider in one paper.


\section{Conclusion}
\textbf{Virtual reality games can and will undergo a lot of changes, especially with the general development of VR in the near future. It will, however, take time, money, and a combined effort on the part of many people.}

VR and HMD are not yet core technologies in todays gaming society but I am being confident that this will change rapidly and the advantages will quickly adapt to many areas of our every day life, as well as gaming.

Almost all of the senses of the human body can be reached with todays technology. Some people may like the thought of totally immersed gaming, other find it disturbing.\newline
Although a controversy appears in some fields the research will surely go on in the next years bringing many innovative and tense devices to our lives including better interaction, conclusive visual effects, stunning audio and astonishing touchable illusions. 

There are just many ways of finding new ways to improve VR gaming in the future. VR is on the edge of becoming a main way to experience games and the enormous potential offers just too many aspects to consider in one paper.


% Referenzen
% References must be the same font size as other body text.
\bibliographystyle{SIGCHI-Reference-Format}
\bibliography{citation}



topics: 
\begin{itemize}
	\item movement: 
	\begin{itemize}
		\item point and teleport
		\item walk (with moving floor tiles?)
		\item player does not move, but rather the environment around him (car, train, airplane)
	\end{itemize}
	\item general input: 
	\begin{itemize}
		\item controller
		\item sound
		\item photo electric effects
	\end{itemize}
	\item health: 
	\begin{itemize}
		\item cybersickness
		\item psychological profits (people with psycological(?) illness can be treated)
		\item psychological disadvantages (people can confuse VR with reality)
	\end{itemize}
	\item other senses
	\begin{itemize}
		\item feel (e.g. something like the iDummy)
		\item taste (Virtual Sweet)
		\item hear
	\end{itemize}
	\item electrical muscle stimulation
	\item electrical nerve stimulation
	\item motivation
	\begin{itemize}
		\item action
		\item social
		\item mastery
		\item achievement
		\item immersion
		\item creativity
		\item \textbf{age}
	\end{itemize}
	\item technical limitations
	\begin{itemize}
		\item graphics
		\item item count
	\end{itemize}
	\item (social) acceptance
	\item portability
	\item displaying the player in the game (multiplayer stuff)
\end{itemize}

\end{document}

%%% Local Variables:
%%% mode: latex
%%% TeX-master: t
%%% End:
